% LaTeX Curriculum Vitae Template
%
% Copyright (C) 2004-2009 Jason Blevins <jrblevin@sdf.lonestar.org>
% http://jblevins.org/projects/cv-template/
%
% You may use use this document as a template to create your own CV
% and you may redistribute the source code freely. No attribution is
% required in any resulting documents. I do ask that you please leave
% this notice and the above URL in the source code if you choose to
% redistribute this file.

\documentclass[letterpaper]{article}

\usepackage{hyperref}
\usepackage{geometry}
\usepackage{mhchem}

% Comment the following lines to use the default Computer Modern font
% instead of the Palatino font provided by the mathpazo package.
% Remove the 'osf' bit if you don't like the old style figures.
%\usepackage[T1]{fontenc}
%\usepackage[sc,osf]{mathpazo}

% Set your name here
\def\name{Kannan K. Puthuval}

% Replace this with a link to your CV if you like, or set it empty
% (as in \def\footerlink{}) to remove the link in the footer:
\def\footerlink{}

% The following metadata will show up in the PDF properties
\hypersetup{
  colorlinks = false,
  urlcolor = blue,
  pdfauthor = {\name},
  pdfkeywords = {
    software,
    product,
    research,
    smart,
    watering,
    water,
    efficient,
    efficiency,
    irrigation,
    home,
    automation,
    soil,
    moisture,
    connected,
    yard,
    internet,
    of,
    things,
    IoT,
    python,
    django,
    cloud,
    application,
    devops,
    computer,
    computing,
    CAD,
    design,
    biology,
    plant,
    agriculture,
    ecology,
    climate,
    change,
    bioenergy,
    energy,
    sustainable,
    sustainability,
    engineering,
    engineer,
    developer,
    development,
    resume,
    CV,
    C.V.,
    },
  pdftitle = {\name},
  pdfsubject = {\name: Software & Product Engineer},
  pdfpagemode = UseNone
}

\geometry{
  body={6.5in, 8.5in},
  left=1.0in,
  top=1.25in
}

% Customize page headers
\pagestyle{myheadings}
\markright{\name}
\thispagestyle{empty}

% Custom section fonts
\usepackage{sectsty}
\sectionfont{\rmfamily\mdseries\Large}
\subsectionfont{\rmfamily\mdseries\itshape\large}

% Other possible font commands include:
% \ttfamily for teletype,
% \sffamily for sans serif,
% \bfseries for bold,
% \scshape for small caps,
% \normalsize, \large, \Large, \LARGE sizes.

% Don't indent paragraphs.
\setlength\parindent{0em}

% % Make lists without bullets
% \renewenvironment{itemize}{
%   \begin{list}{}{
%     \setlength{\leftmargin}{1.5em}
%   }
% }{
%   \end{list}
% }

\begin{document}

% Place name at left
{\huge \name}

% Alternatively, print name centered and bold:
% \centerline{\huge \bf \name}

\vspace{0.25in}

\begin{minipage}{0.45\linewidth}
  3400 Avenue of the Arts, Apt J222 \\
  Costa Mesa, CA 92626 \\
\end{minipage}
\begin{minipage}{0.45\linewidth}
  Phone: (312) 600-4057 \\
  Email: \href{mailto:kannan.puthuval@gmail.com}{kannan.puthuval@gmail.com} \\
\end{minipage}

\section*{Experience}

  \subsection*{Software Engineer, The Scotts Company, 2017 - present.}

    \begin{itemize}

      \item Managed the release and deployment of an IoT cloud application serving 15k users using Python, Django, Docker, AWS-EC2, AWS-ECS, AWS-S3.
      \item Led maintenance and upgrades for CI/CD pipeline using Travis, Jenkins, GitHub, and SaltStack.
      \item Led the major rewrite of an embedded IoT gateway application, using Python and embedded Linux.
      \item Led the mechanical engineering and new product development of an IoT irrigation timer.
      \item Developed software and hardware of a data acquisition system for measuring flow meter performance.
      \item Developed a framework for continuous automated testing of IoT devices.
      \item Developed a BLE interface for factory testing of an IoT device.
      \item Debugged and analyzed distributed system of cloud, mobile, and embedded components.

    \end{itemize}

  \subsection*{Chief Scientist \& Director of Product Development, Oso Technologies, 2015 - 2017.}

    \begin{itemize}

      \item Directed the development of four new hardware products for monitoring and caring for plants; a smart soil moisture sensor, a smart watering valve, and two ZigBee networking devices.
      \item Developed technology for capacitive soil moisture sensing and implemented in a new device.
      \item Designed, prototyped and tested injection-molded plastic enclosures to IP67 standards using CAD software, 3D printing and mold casting.
      \item Designed new test protocols, conducted experiments and analyzed data to characterize the behavior of existing soil moisture sensors in various soil types.
      \item Built software tools to improve the customer support experience by analyzing and visualizing data and automating support agent tasks.

    \end{itemize}

  \subsection*{Research Coordinator, University of Illinois SoyFACE Project, 2012 - 2015.}

    \begin{itemize}

      \item Directed field research operations and coordinated a multi-disciplinary team of researchers and technicians at SoyFACE, a field site investigating the interaction of plants and global change.
      \item Developed, implemented and maintained new research tools including data acquisition and control software, free-air \ce{CO2} and \ce{O3} enrichment systems, and a system of automated and instrumented rainout shelters for an open-field drought experiment.
      \item Curated site-generated data and developed GIS for infrastructure and experiment planning.

    \end{itemize}

  \subsection*{Research Technician, University of Illinois SoyFACE Project, 2009 - 2012.}

    \begin{itemize}

      \item Operated and maintained equipment, including rainout shelters, free-air concentration enrichment systems, and environmental growth chambers.
      \item Collected and analyzed data on weather,soil, plant physiology, and gene expression.

    \end{itemize}

\section*{Patents}

  \begin{itemize}
    \item Mane, M., Singer, D., Puthuval, K. (2018) {\it USD829574S1} Retrieved from \href{http://patft1.uspto.gov/netacgi/nph-Parser?patentnumber=D829574}{http://patft1.uspto.gov/netacgi/nph-Parser?patentnumber=D829574}
  \end{itemize}

\section*{Publications}

  \begin{itemize}

    \item Gray, S.B., Strellner, R.S., Puthuval, K.K., Ng, C., Shulman, R.E., Siebers, M.H., Rogers, A., and Leakey, A.D.B. (2013). Minirhizotron imaging reveals that nodulation of field-grown soybean is enhanced by free-air \ce{CO2} enrichment only when combined with drought stress. {\it Functional Plant Biology} 40, 137-147.
    \item Gray, S.B., Strellner, R.S., Puthuval, K.K., and Leakey, A.D.B. (2011). Elevated \ce{CO2} increases stomatal closure under reduced soil moisture in soybean {\it(Glycine max)}. {\it American Society of Plant Biologists Annual Meeting}. Minneapolis, MN.
    \item Gray, S.B., Strellner, R.S., Puthuval, K., and Leakey, A.D.B. (2010). Free-air \ce{CO2} enrichment does not lessen the impact of drought on soybean photosynthesis under field conditions. {\it The Ecological Society of America 95\textsuperscript{th} Annual Meeting}. Pittsburgh, PA.

  \end{itemize}

\section*{Education}

  \begin{itemize}
    \item B.S. Integrative Biology, University of Illinois, 2007
  \end{itemize}

\section*{Service \& Other Experience}

  \subsection*{Volunteer, Engineers Without Borders UIUC Nigeria Water Project, 2011 - 2012.}

  \subsection*{Foreign English Teacher, Jishou University, Hunan Province, China, 2007 - 2008.}

  \subsection*{Co-Director, The Bike Project of Urbana-Champaign, 2005 - 2007.}

\section*{References}

\begin{itemize}
  \item Anthony Huy, Product Development Lead, The Scotts Company (\href{mailto:anthony.huy@scotts.com}{anthony.huy@scotts.com})
  \item Kaido Kert, Tech Lead Manager, YouTube (\href{mailto:kaidokert@gmail.com}{kaidokert@gmail.com})
  \item Mercedes Mane, COO, The Product Manufactory (\href{mailto:mercedes@theproductmanufactory.com}{mercedes@theproductmanufactory.com})
\end{itemize}

\bigskip

% Footer
\begin{center}
  \begin{footnotesize}
    Last updated: \today \\
    \href{\footerlink}{\texttt{\footerlink}}
  \end{footnotesize}
\end{center}

\end{document}
